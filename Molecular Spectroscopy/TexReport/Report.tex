\documentclass[reprint,amsmath,amssymb,aps, prl]{revtex4-2}
\usepackage{graphicx}% Include figure files
\usepackage{dcolumn}% Align table columns on decimal point
\usepackage{bm}% bold math
\usepackage{amsmath}
\usepackage{natbib}
\usepackage{chngcntr}
\usepackage{url}

\bibliographystyle{apsrev}
\title{Bibliography management: \texttt{natbib} package}
\setcounter{tocdepth}{5}
\setcounter{secnumdepth}{5}

\begin{document}

\preprint{APS/123-QED}

\title{Manuscript Title:\\with Forced Linebreak}% Force line breaks with \\


\author{Alon Shaaltiel}
\author{Oren Kereth}
\affiliation{Raymond and Beverly Sackler School of Physics and Astronomy, Faculty of Exact Sciences, Tel Aviv University, Tel Aviv 69978, Israel}



\date{\today}% It is always \today, today,
             %  but any date may be explicitly specified

\begin{abstract}
An article usually includes an abstract, a concise summary of the work
covered at length in the main body of the article. 
\begin{description}
\item[Usage]
Secondary publications and information retrieval purposes.
\item[Structure]
You may use the \texttt{description} environment to structure your abstract;
use the optional argument of the \verb+\item+ command to give the category of each item. 
\end{description}
\end{abstract}

%\keywords{Suggested keywords}%Use showkeys class option if keyword
                              %display desired
\maketitle
\renewcommand*{\thesection}{\arabic{section}}
\renewcommand*{\thesubsection}{\arabic{section}.\arabic{subsection}}
% \renewcommand{\thefigure}{\arabic{figure}}

\section{Introduction}
In this experiment the Fourier Transform Infra Red spectrometer (FTIR) will be used extensively to extract the infra-red spectrum of various materials and compounds. The infra-red spectrum consists of the intensity of incoming light made out of a range of wavelengths transmitted through a sample. The sample absorbs some wavelengths thus yielding an IR spectrum of that particular sample.
% We never said what IR is outside of FTIR. re: added

A short historical overview is presented for the reader, for more information see References.
% Change to "A short historical overview is presented for the reader, for more information see References." re: changed
In 1800 IR was recognized as a distinct region of the energy spectrum by Sir William Herschel, however the study of IR light with various materials started about 100 years later. In 1903, William W.Coblentz measured the IR spectra of hundreds of inorganic and organic compounds.
% Why did he measure them? re: no idea, wasn't written
As a result of increasing interest in the field, the first prototypes of IR spectroscopy were built in the 1930s.  In 1949 the astrophysicist Peter Fellgett used an interferometer to measure light from celestial bodies and produced the first Fourier transform infrared spectrum, however this method was still limited and took many hours to compute. In the late 1960s commercial FTIR spectrometers appeared as
% because sounds less scientific than as. re: ok
microcomputers were able to do the Fourier transform, however
% maybe say "contrary to what the name suggests, they were..." re: i don't know about that
they were large and expensive. Over time, technology reduced the costs and size,
% size? re: ok
increased availability and enhanced the capabilities of FTIR spectroscopy systems\cite{spechistory}


\subsection{Thin Film Interference}
Upon contact with a beam of light, the internal reflections in a thin sample of material with a sufficiently high refractive index are not negligible.
\begin{figure}[h]
    \includegraphics[width=\linewidth]{Images/Thin film.png}
    \caption{Reflected and transmitted light waves from a thin film}
    \label{fig:ThinFilm}
\end{figure}
This experiment concerns only the transmitted beams; The treatment for reflected beams is quite similar.\\
In this experiment only the intensity of the beams is detected, to understand how coherent light beams with the same wave-length (and frequency) affect the detected intensity an example of two beams is given. 
The Electric fields of the two beams are
\begin{equation*}
    \vec{E_{i}}=E_{i}e^{i(\omega t - \vec{k}\cdot\vec{r} + \varphi_i)}\hat{\epsilon_1};\ i=1,2
\end{equation*}
% Should I write what everything represents?
The detectable wave is the super-position of both waves
\begin{equation*}
    \vec{E} = (E_{1}+E_{2}e^{i\Delta\varphi})e^{i(\omega t - \vec{k}\cdot\vec{r} + \varphi_1)}\hat{\epsilon_1};\ \Delta\varphi=\varphi_2-\varphi_1
\end{equation*}
An Electromagnetic wave's intensity is proportional to $|\vec{E}|^2$ thus, the detected intensity is proportional to 
\begin{equation*}
\begin{split}
    |\vec{E}|^2 = E_{1}^2 + E_2^2 + \Re\{2E_{1}E_{2}e^{i\Delta\varphi}\}\\ \rightarrow I=I_1+I_2+2\sqrt{I_1I_2}cos(\Delta\varphi)
\end{split}
\end{equation*}
If $\Delta\varphi=2n\pi;\ (n\in\mathbb{Z})$, $I$ is at its maximum, this is called "Constructive interference". If $\Delta\varphi=(2n+1)\pi;\ (n\in\mathbb{Z})$, $I$ is at its minimum, this is called "Destructive interference". If $\Delta\varphi$ is changed linearly the detected intensity would result in a cosine$\backslash$sine.\\
Now onto the transmitted beams, the phase between two consecutive transmitted beams and relative intensity are
\begin{equation} \label{eq:ThinPhaseAmp}
\Delta\varphi=k(2n_{2}dcos\theta);\frac{I_{n+1}}{I_{n}}=r^{2}
\end{equation}
where $k=\frac{2\pi}{\lambda}$ is the beam's wave number, $n_{2}$ is the sample's refractive index, $\theta$ is the beam's angle inside the sample,  d is the sample's thickness, $I_{n}$ is the n-th transmitted beam's intensity and $r<1$ is the reflection coefficient of the material. $2n_{2}dcos\theta$ is the Optical Path Length (OPL) difference and measures the difference in $\vec{r}$ the latter beam gained. Constructive interference occurs if $\text{OPL}=\frac{m}{k}$ and if $\text{OPL}=\frac{m+\frac{1}{2}}{k}$, destructive interference occurs. If k were changed continuously, the difference between two constructive peaks of intensity would be
\begin{equation} \label{eq:ThinDiffWavenum}
\Delta k=\frac{1}{2n_{2}dcos(\theta)}
\end{equation}

% By measuring both $\Delta k$ using a Fourier transform of the intensity or a sine fit, and the thickness of the sample in the lab, the material's refractive index can be extracted 
% \begin{equation} \label{eq:ThinRefractIndex}
% n_2=\frac{1}{2\Delta kdcos(\theta)}
% \end{equation}

\subsection{The Beer-Lambert Law}
The Beer-Lambert Law describes how a monochromatic beam of light is absorbed in relation to substance concentration. The dependency is given by
\begin{equation} \label{eq:BeerLambert}
ln\left(\frac{P(v)}{P_{0}(v)}\right)=A=-av_{\alpha\alpha'}Cx
\end{equation}
where $P(v)$ is the light's intensity through the sample, $P_{0}(v)$ is the intensity with no sample, $A$ is the absorbance, $a$ is the molar attenuation coefficient, $\alpha$ and $\alpha'$ are quantum numbers, $C$ is the concentration, and $x$ is the optical path length of the beam, given by $n\cdot l$ where $l$ is the distance the beam traveled and $n$ is the material's refractive index. All else being equal, the absorbance and concentration are proportional to one another. \\
The Beer-Lambert Law does have its limitations, for starters, it is only valid for low concentrations, according to (\Citeauthor{2020Beer}, \cite{2020Beer}) at higher concentrations "the individual particles of analyte no longer are independent of each other", this may change the material's absorptivity. On top of that, at high concentrations a solution's refractive index may change across concentrations, this too may change a material's refractive index \cite{2020Beer}.

\subsection{CO Molecule IR Spectrum}
The model used to describe CO molecules
% either write "a CO molecule" or just shorten to "The model used to describe diatomic molecules..." re: agreed
and other diatomic molecules pictures a molecule in which the individual atoms, held together by chemical bonds, are in vibratory motion along these bonds, while the molecule as a whole is rotating \cite{alpert}. The energy states of this system correspond to a Hamiltonian with a potential 
% maybe add a H = ?? + V; V=... to the equation to make it more readable? re: I'm not entirely sure.
\begin{equation} \label{eq:CO_Potential}
V(\boldsymbol{r})=\frac{1}{2}k(r-r_0)^2
\end{equation}
and can be expressed as a combination of rotational and vibrational energies
\begin{equation} \label{eq:CO_RotVibEnrg}
E=E_{Rot}+E_{Vib}= hcw_{e}(\nu +\frac{1}{2}) + \frac{\hbar  ^2}{2I}J(J+1)
\end{equation}
where $h$ is Planck's constant, $c$ is the speed of light, $w_{e}$ is the wave number corresponding to the frequency of vibrations, $\nu$ is a quantum number corresponding to the vibrations,  $I$ is the molecule's moment of inertia and $J$ is a quantum number corresponding to the angular momentum of the particle \cite{griffithsQM}. However, the  current potential is not truly harmonic and only serves as an approximation,
% "However, this is only an approximation" may sound better re: integrated it into the text now
as effects such as the centrifugal force, Coriolis and Fermi \cite{alpert} require the use of perturbation theory \cite{samurai}. Using the perturbation theory, the energy states are now
% now is unnecessary IMO re: Wanted to elaborate a little more and also add a reference.
\begin{equation} \label{eq:CO_PeturbRotVibEnrg}
\begin{split}
& T=T_{Vib}+T_{Rot}=\\
& w_{e}(\nu +\frac{1}{2}) -w_{e}x_{e}(\nu +\frac{1}{2})^2+ \\
& B_{v}J(J+1)-D_{v}J^2(J+1)^2 
\end{split}
\end{equation}
where $x_{e}$, $B_{v}$ and $D_{v}$ are constants which take into account small perturbations and $T$ is the energy states in wave number units ($T=\frac{E}{hc}$
% isn't it h-bar? re: no, in theory its h
). At room temperature only the $\nu=0$ state is occupied while the $J$ states are all occupied. Therefore, using selection rules \cite{griffithsQM} only the transitions $\nu$ $\rightarrow $ $\nu'$  where $\nu ' =1,2,3$  and $J$ $\rightarrow $ $J'$ where $J'=J+1,J-1$   are of interest. Using equation \ref{eq:CO_PeturbRotVibEnrg}
% notice  re: thanks, forgot about it                ^^^^^^^^^^^^^^^^^^^^^^^^^^^^^^^^^^^^^^^^^^^^^^^^^^^
the allowed spectral lines are derived to be
\begin{equation} \label{eq:CO_DiffPeturbEnrg}
\begin{split}
& \Delta T =T(\nu ',J')-T(0,J)=\\
& w_{e}\nu' -w_{e}x_{e}(\nu'^{2})+(2B_{e}-\alpha(\nu'+1))m-\\
& \alpha\nu'm^2-4D_{v}m^{3}
\end{split}
\end{equation}
where $\Delta T$ is the allowed spectral line, $B_{e},\alpha $ are constants related to $B_{v}$  which are the result of further corrections due to perturbation theory and $m$ is either $J+1$ or $-(J-1)$  depending on the transition. In this experiment the transitions to $\nu'=1$ and $\nu' =2$ will be of interest, for which $\Delta T=2141cm^{-1}$ and $\Delta T = 4255 cm^{-1}$ respectively. (maybe put a reference, maybe it's in STUART)


\section{Experimental System and Measurements}
There are two main measuring instruments in this experiment- FTIR and ATR. Both use Michelson interferometers and perform a Fourier transform.

\subsection{Michelson Interferometers}
A Michelson interferometer consists of two perpendicularly plane mirrors, one of which can travel in a direction perpendicular to the plane.
% maybe say "can travel in and out" while less scientific per se, it's more clear. RE: I agree that it can be more clear that way, but the figure helps in understanding it. if you think otherwise I'll change it. 
A semi reflecting film, the beamsplitter, bisects the planes of these two mirrors. The beamsplitter splits the beam into two, one beam transmitted through the beamsplitter and the other reflected from it. Both beams then get reflected from the mirrors, returning to the beamsplitter where they recombine and interfere \cite{stuart} (see Figure \ref{fig:Interferometer}). 
% needs reference re: fixed
\begin{figure}[h]
    \includegraphics[width=7cm]{Images/INTERFEROMETER.png}
    \caption{Schematic of a Michelson interferometer \cite{stuart}.}
    \label{fig:Interferometer}
    \centering
\end{figure}
The beam which emerges from the interferometer at 90 degrees to the input beam is called the transmitted beam and is the beam detected by the detector as can be seen in Figure 2.
% "the beam detected by the detector" to "the detected beam" re: this was the first time I mentioned the detector.
The moving mirror produces an optical path difference between the two split beams.
% "between the two beams split by the beamsplitter" to "between the two split beams" re: fixed
Plotting the signal produced by the detected beam
% maybe change to "the beam's intensity" re: i don't know, signal sounds more intuitive 
as a function of the change of
% (optical)
optical path length between the two beams yields
% "will yield" to "yields" ? re: fixed
an interferogram. By performing a Fourier transform on the interferogram the spectral distribution of the beam (i.e. the amplitude of each wave number the light is made of.)
% "is consisted of" to either "consists of" or "is made of re: fixed
can be found.

\subsection{FTIR}
FTIR stands for Fourier Transform Infra-Red spectrometer.
% (spectrometer) and maybe capitalise the T and I, maybe take "infrared" to "Infra Red" to make it even clearer re: fixed
IR radiation is passed through a sample.
Some
% I think LTex is right, prolly remove the "Some of" re:fixed 
infrared radiation is absorbed by the sample and some of it is
passed through (transmitted). The resulting spectrum represents the molecular
absorption and transmission \cite{FTIRmanual}. FTIR (see Figure 3) utilizes Michelson interferometers, the interferogram and a Fourier transform to perform the analysis of the spectrum. 
\begin{figure}[h]
    \includegraphics[width=7cm]{Images/FTIR LAYOUT.png}
    \caption{Layout of the FTIR spectrometer \cite{FTIRmanual}.}
    \label{fig:FTIR}
    \centering
\end{figure}

The main components of the FTIR spectrometer are
% maybe add : or "as follows" re: i don't think its needed
the source, a black-body source from which the IR radiation is emitted. The radiation then passes through an interferometer, where the 'spacial encoding' takes place \cite{FTIRmanual}. A He-Ne laser is used as a reference for calibration of the moving mirror's position \cite{stuart}. After going through the interferometer, an interferogram is created. The interferogram then passes through the sample, where specific wave numbers are absorbed according to the sample's characteristics and the rest are transmitted into the detector where the interferogram signal is measured. The data reaches the computer, which performs a Fourier transform yielding the infrared spectrum consisting of absorption lines corresponding to the sample. 

\subsection{ATR}
ATR, which stands for \emph{Attenuated Total Reflection}, measures the absorbance of a liquid or solid using IR spectroscopy. A crystal with a sufficiently high refractive index to achieve total internal reflection is surrounded by the material whose absorbance is of interest. 
\begin{figure}[h]
    \includegraphics[width=7cm]{Images/ATR.png}
    \caption{Layout of an ATR compartment \cite{wiki:ATR}}
    \label{fig:ATR}
    \centering
\end{figure}
An interferogram is projected into the crystal and evanescent waves, resulting from the total internal reflection, propagate through the sample. 
Evanescent waves, similarly to most waves are a product of an amplitude and an exponent: $|\vec{E}|=Ae^{i(\omega t - \vec{k}\cdot\vec{r})}$, however, unlike most EM waves, evanescent waves have a complex $\omega$ (and $|\vec{k}|$). Separating $\omega$ and $\vec{k}$ to their Real and Imaginary parts and choosing $\vec{k}$'s direction as $\hat{z}$ yields
\begin{equation}
    |\vec{E}|=Ae^{i(\omega_R t - k_{R}z)}e^{-(\omega_I t + k_{I}z)}
\end{equation}
This results in an exponentially decreasing amplitude and intensity.
In a vacuum, the energy evanescent waves carry is wholly returned $\langle\frac{dU_{Eva}}{dt}\rangle=0$. However, if a sample is inserted into the ATR, and evanescent waves propagate through it, wave-lengths corresponding with the sample's normal modes are absorbed and the energy of the reflected beam is decreased $\langle\frac{dU_{Eva}}{dt}\rangle<0$. As described in the previous paragraph, the interferogram is detected and disassembled into the sample's infrared spectrum.

\subsection{Transmission Through Thin Films of Polyester}
In this part of the experiment the refractive index of a thin sample of Polyester is extracted. Thin Polyester samples with varying widths are placed in the sample area perpendicular to the beam's direction. An interferogram is projected through each sample. The resulting IR absorbance spectrum will be a summation of the material's (Polyester's) absorbance spectrum and the sine wave showing absorption from thin film interference. To extract the frequency of interference ($\frac{1}{\Delta k}$ in eq. \ref{eq:ThinDiffWavenum}) a sine wave is fitted to the region in the absorbance spectrum with the clearest sine wave, the fit's frequency is taken as the sample's frequency. Using equation \ref{eq:ThinDiffWavenum}, the extracted sine frequency, the measured width, and $\theta=\frac{\pi}{2}$, as the sample is perpendicular to the beam, a fit of the form
\begin{equation}
    \begin{gathered}
        \label{eq:ThinLinFit}
        d=a_{0}+a_{1}f_{fit} \\
        f_{fit} = \frac{1}{\Delta k};\ a_1=2n_{2};\ a_0\approx 0
    \end{gathered}
\end{equation}
was made. Some samples' widths were not measured and only had an interferogram projected through them. By fitting a sine to their IR spectra and inputting their fitted frequency into the calibrated relation in equation \ref{eq:ThinLinFit}, their widths can be approximated.

\subsection{Calibration Using Ethanol} 
In this part of the experiment the relationship between the concentration of a dissolved substance inside a solution
% , re: where?
to its total absorbance amplitude, which is assumed to be linear according to equation \ref{eq:BeerLambert} was calibrated. The calibration was done using mixtures of water and ethanol  with varying known concentrations. The mixture was put into a tank inside an ATR. Starting from a concentration of $96\%$ , the ethanol was diluted by adding more water into the mixture, thus changing the concentration of the ethanol according to 
\begin{equation} \label{eq:EthanolConcentrVolume}
C_{i}V_{i} = C_{f}V_{f}
\end{equation}
where $C_{i}$,$V_{i}$, $C_{f}$,$V_{f}$ are the initial and final concentration of ethanol and volume of the mixture respectively. 
For each concentration, $F$ , the total absorbance of the solution in the range $1000-1120 cm^{-1}$ was calculated. In this range lie the absorption peaks of ethanol \cite{NISTwebook} and therefore, according to \ref{eq:BeerLambert} their intensity is proportional to the concentration of ethanol in the mixture. A calibration graph of the form 
\begin{equation} \label{eq:EthanolLinFit}
    C=a_{0}+a_{1}F 
\end{equation} 
was then made, Where C is the concentration of ethanol in the mixture. Using the calibration, the concentration of different solutions was found. 

\subsection{$\text{CO}_{2}$ and IR Spectrum of CO}
Using a pump, exhaled and outside air were each inserted into a chamber in the FTIR. For each of the two types of air the total absorbance in the IR spectrum in the range $2270-2400cm^{-1}$ was
% was re: fixed
calculated. In this region there are two absorption peaks both corresponding to the concentration of $\text{CO}_{2}$ in the air. The ratio between the total absorbance of the exhaled air to the outside air in that region, denoted by $S$ is therefore equal to the corresponding ratio between $\text{CO}_{2}$ concentrations in each of them. $S$ will then be compared to previously measured ratios. 
% "ratios previously measured" to "previously measured ratios" re: fixed
Afterwards, by pumping CO into to the same chamber in the FTIR, its IR spectrum was measured.
% I don't think "and" is supposed to be there re: fixed
According to equation \ref{eq:CO_DiffPeturbEnrg}
% need to reference re: fixed
the spectrum contains absorption peaks that represent a transition in the angular momentum of the molecule, $J$  to either $J+1$ or $J-1$ (see Figure \ref{fig:CoEnergy})
% need to reference re: fixed
\begin{figure}[h]
    \includegraphics[width=7cm]{Images/COEnergystates.jpg}
    \caption{CO absorption lines in the region $2025-2033 cm^{-1}$. Each peak represents a transition $m$ of the angular momentum and the transition from $\nu=0$ to $\nu'=1$ which is centered at 2142$cm^{-1}$  as can be seen and according to subsection 1.3.
    % need to reference
    }
    \label{fig:CoEnergy}
    \centering
\end{figure}
Each of the peaks, $m$ , were numbered according to its order relative to the center point at 2142$cm^{-1}$ (i.e. if $m$ was the first to its left
% what's "its" (I know you're talking about the center of the mustache but it's pretty unclear), writing "to the left" instead of "to its left" is also clearer if you'd prefer that
it was numbered as $-1$ and if it was the first to its right
% same as last comment
it was numbered as $1$). Afterwards, a polynomial fit of $3^{rd}$ order was made between $\Delta T$ from equation 8 
% need to reference
and $m$
\begin{equation} \label{eq:CO_EnrgStatesFit}
    \Delta T = a_{0}+a_{1}m+a_{2}m^2+a_{3}m^3
\end{equation}
The coefficients extracted from the fit will then be used to extract the coefficients in equation \ref{eq:CO_PeturbRotVibEnrg}
% need to reference re: fixed
according to the relations 
\begin{equation}
\begin{aligned}
\alpha &=-a_{2} & D_{v}&=-\frac{a_{3}}{4} & w_{e}&=2a_{0}-\frac{4255}{2} \\
B_{e}&=\frac{a_{1}-2a_{2}}{2} &  x_{e}&=\frac{a_{0}-\frac{4255}{2}}{2a_{0}-\frac{4255}{2}}
\end{aligned}
\label{eq:COFITcoefficients}
\centering
\end{equation}
where in order to find $w_{e}$ and $x_{e}$ the spectral line of the transition $\nu =0 \rightarrow \nu'=2$ and $\Delta J =0$ was used. 

\subsection{IR Spectra of Different Cigar Brands}
In this part of the experiment gasses in the smoke of various cigars are analyzed. An interferogram is projected through regular air. A pump smokes on a cigar and fills the sample area with smoke. An interferogram is projected through the smoke and the absorbance spectrum compared to regular air is created. Various toxic gasses are identified. Each gas's total absorption (TA) was calculated using trapezoidal rule \cite{numerical} and taken as a proxy to concentration. The TA of different substances cannot be compared; However, The TA of the same substance in different cigar brands can be directly compared. The comparison will look at the different substances and relative quantities and try to rank which cigar is least to most poisonous.
% Maybe correct the concentration claim

\section{Results}

\subsection{Polyester Refractive Index Calculation}
For each sample width the area with the clearest sine wave in the absorption spectrum was found and fitted to a sine wave.
\begin{figure}[h]
    
    \includegraphics[width=\linewidth]{Images/29_2_Spec.png}
    \caption{IR absorbance spectrum and sine fit to the selected region for a width of $29.2\ [\mu m]$}
    \label{fig:SineFitEx}
    \centering
\end{figure}
Fitting a linear function between the width and the frequency of the associated sine (eq. \ref{eq:ThinLinFit}) yielded the following (See Figure \ref{fig:ThinLinWidthFreq})
\begin{figure}[h]
    \includegraphics[width=\linewidth]{Images/__linear_fit.png}
    \caption{Linear fit between the fitted frequency, $f_{fit}$, and the sample width.}
    \label{fig:ThinLinWidthFreq}
    \centering
\end{figure}
The fit parameters of the linear fit can be seen in Table \ref{tbl:WidthFit}
\begin{table}[h]
    \begin{tabular}{ |p{2cm}|p{2cm}|p{3.3cm}|  }
     \hline
     \multicolumn{3}{|c|}{Fit Parameters of Width vs. Frequency} \\ \hline
     Parameter & Value & Error (Relative Error)\\ \hline
     $a_{0}$[$cm$]  &$2.4$      &$1.5$ ($60\%$) \\ 
     $a_{1}$[$1$]   &$3.080$    &$0.060$ ($1.9\%$)  \\ \hline
    \end{tabular}
    \caption{Fit parameters of the sample width vs fitted frequency graph}
    \label{tbl:WidthFit}
\end{table}

Using equation \ref{eq:ThinDiffWavenum}, the extracted refractive index is
\begin{equation}
    n = \frac{a_1}{2} = 1.540 \pm 0.030 [1]
\end{equation}
Which is in the general area of plastics' refractive indices \cite{plasticRefract}. This is an average over the range of wave lengths in the interferogram and so, pinpointing which plastic in particular is used is impossible from the refractive index alone. Using the IR spectrum to further narrow down options may be possible, but is out of the scope of this paper.
Using the calibrated correlation between sample width and sine frequency, the widths of three samples, whose widths were not measured but had an interferogram projected through them were approximated and can be seen in table \ref{tbl:ThinMissingWidths}.
\begin{table}[h]
    \begin{tabular}{ |p{2.3cm}|p{3cm}|p{3cm}|  }
     \hline
     \multicolumn{3}{|c|}{Fit Parameters of Width vs. Frequency} \\ \hline
     Filename & Frequency & Width\\ \hline
     Unknown  &$122.85\pm0.26$ [$cm$]    &$ 39.10\pm0.90$ [$cm$]\\ 
     Unknown E &$163.02\pm0.74$ [$cm$]   &$52.1\pm1.1$ [$cm$]\\ 
     Unknown cpp-c&$142.24\pm0.51$ [$cm$]&$45.4\pm1.0$ [$cm$]\\ \hline
    \end{tabular}
    \caption{Calculated missing sample widths using the calibration between fitted frequency and sample width}
    \label{tbl:ThinMissingWidths}
\end{table}

\subsection{Ethanol Concentration Calibration}
The total absorbance amplitude in the range $1000-1120 cm^{-1}$ was calculated numerically
% numerically calculated fixed:
for each concentration of ethanol in the mixture using the trapezoidal rule\cite{numerical}.
% wikipedia calls it "trapezoidal rule" re: fixed
Plotting the IR spectrum of all the mixtures in that range
% in range sounds better re: not sure 
on the same graph (see Figure \ref{fig:EthanolMixtures})
% need to reference re: fixed
shows the increasing intensity of the peaks for higher and higher concentrations of ethanol. 
\begin{figure}[h]
    \includegraphics[width=7cm]{Images/EthanolAbsorption.jpg}
    \caption{IR spectrum of mixtures with different ethanol concentration in the range 1000-1120$cm^{-1}$ }
    \label{fig:EthanolMixtures}
    \centering
\end{figure}

Fitting a linear function according to equation \ref{eq:EthanolLinFit} to the measurements of concentration and absorbance yielded the following (see Figure \ref{fig:EthanolCalibration}).
% need to reference re: fixed
    \begin{figure}[h]
    \includegraphics[width=7cm]{Images/linear_fittingethanol}
    \caption{Linear calibration graph of concentration of ethanol in the mixture to total absorbance amplitude in the range $1000-1120 cm^{-1}$}
    \label{fig:EthanolCalibration}
    \centering
\end{figure}
The fit parameters extracted from the linear fit can be seen in Table \ref{ethanol table}
% need to reference re:fixed
\begin{table}[h]
\begin{tabular}{ |p{2cm}||p{2cm}|p{2cm}|  }
 \hline
 \multicolumn{3}{|c|}{Fit Parameters of Ethanol Calibration Graph} \\ \hline
 Parameter& Value &Error (Relative Error)\\ \hline
 $a_{0}$[$1$]    &$11.7$    &$9.3$($79\%$) \\
 $a_{1}$[$cm$] &$7.3$    & $1.4$ ( $19\%$)  \\ \hline
\end{tabular}
\caption{Fit parameters of the ethanol concentration vs. absorption amplitude graph}
\label{ethanol table}
\end{table}
The concentration of ethanol
% Capital E %RE: doesn't have to be capitalized actually. for example in wiki it isn't capitalized mid sentence.
in other mixtures was then found by numerically calculating the total absorption in the relevant range and plugging it in the linear equation using the fit parameters. Using the fit parameters from Table \ref{ethanol table} the concentrations in a Suktinis and two unknown alcoholic beverages were found (see Table \ref{beverages table}).
% need to reference tables RE: fixed.
% "Using the fit parameters" is repeated, I think "Table \ref{} shows the calculated concentrations in a Suktinis and two unknown alcoholic beverages", we already said how they were calculated 
%RE: it was explained really briefly in the section regarding ethanol before it and so I wanted to elaborate a bit more.

\begin{table}[h]
\begin{tabular}{ |p{2.8cm}|p{2.8cm}|p{2.8cm}|  }
 \hline
 \multicolumn{3}{|c|}{Ethanol Concentration in Different Beverages} \\
 \hline
 Beverage& Concentration(\%) &Concentration Error (Relative Error) \\ \hline
 Sukinis    &$54.3$   &$8.3$ ($15\%$) \\
 Unknown 1  &$40.5$   &$5.7$ ($14\%$) \\
 Unknown 2  &$72$     & $12$ ($17\%$) \\ \hline
\end{tabular}
\caption{Ethanol concentration in different beverages}
\label{beverages table}
\end{table}

\subsection{$\text{CO}_{2}$}
Using the trapezoidal rule
% rule? RE: fixed
 the total absorbance of the air in the region $2270-2400cm^{-1}$ was calculated.
  The total absorbance in that region is proportional to the concentration of $\text{CO}_{2}$ as there are two peaks correlating to $\text{CO}_{2}$ in said region \cite{NISTwebook} and according to equation \ref{eq:BeerLambert} linear relation is assumed.
% Use $\text{CO}_{2}$ for $\text{CO}_{2}$ RE: ok
 The ratio between the total absorbance of exhaled air and outside air in that region, denoted by $S$, was calculated. $S$  also represents the ratio between $\text{CO}_{2}$ as explained above,
% This sentence is very long. break it up into smaller sentences, like:
% "Using the trapezoidal ??, the total absorbance of the air in the region $2270-2400cm^{-1}$ was calculated. As the total absorbance is closely correlated to the concentration of CO2 in the air, a ratio between the total absorbance of exhaled air and outside (or Tel Aviv) air, denoted by $S$, was calculated. $S$ represents the ratio between CO2 in the air and CO2 in exhaled air RE: integrated some of what you said, what do you think? 
\begin{equation}
\label{eq:CO2ratio}
S=\frac{Absorbance_{Exhaled}}{Absorbance_{Outside}}=61 [1] % Perhaps add \frac{Absorbance_{Exhaled}}{Absorbance_{Outside}}. just writing S is a bit unclear RE:fixed
\end{equation}
$S$
% "This value" to $S$ RE: fixed
would later be
% would later be RE: fixed
compared to that measured a few years ago to see the trend line.  
\subsection{CO}
The absorption peaks from the IR spectrum of CO were extracted and the absorption lines,
% ,
$\Delta T$, 
% ,
and their corresponding angular momentum transitions,
% ,
$m$,
% , RE: fixed all of these
were measured relative to the center point at $\Delta T=2141 cm^{-1}$ which represents $\Delta J = m = 0 $ and the vibrational transition from $\nu =0$ to $\nu' =1$. Using the locations of the absorption peaks,
% "Using the locations of the absorption peaks, " is clearer RE:fixed
a $3^{rd}$ order polynomial fit was made according to equation \ref{eq:CO_EnrgStatesFit} (see Figure \ref{fig:COPolynomialfit}).
% needs reference RE: fixed
\begin{figure}[h!]
    \includegraphics[width=7cm]{Images/polynomial_3_fitting.png}
    \caption{$3^{rd}$ order polynomial fit of the spectral lines of the absorption peaks $\Delta T$ as a function of the transition of $CO$'s
    % "the molecule" to "CO" RE: ok
    angular momentum $m$. }
    \label{fig:COPolynomialfit}
    \centering
\end{figure}
The fit parameters extracted can be seen at Table \ref{COfittable}. 
% needs reference RE: fixed
\begin{table}[h]
    \resizebox{9cm}{!}{
    \begin{tabular}{ |p{2.8cm}||p{2.8cm}|p{2.8cm}|  }
    
     \hline
     \multicolumn{3}{|c|}{Fit Parameters of the CO absorption lines $3^{rd}$ polynomial fit} \\
     \hline
     Parameter& Value(\%) & Error (Relative Error) \\ \hline
    $a_{0}[cm^{-1}]$    &$2143.293$  &$0.034$ ($0.002\%$) \\
    $a_{1}[cm^{-1}]$  &$3.8220 $  &$0.0040$  ($0.11\%$) \\
    $a_{2}[cm^{-1}]$  &$-0.01800$     & $1.5e-4$ ($0.83\%$) \\ 
    $a_{3}[cm^{-1}]$  &$-9.4e-6$     & $1.30e-5$ ($140\%$) \\ \hline
    \end{tabular}
    }
    \caption{Fit parameters of the $3^{rd}$ order polynomial fit}
    \label{COfittable}
    \end{table}
Using these fit parameters the coefficients at equation \ref{eq:CO_DiffPeturbEnrg} were extracted via the relations stated at equation \ref{eq:COFITcoefficients} (see Table \ref{COcoeftable}).
% needs reference RE: fixed
\begin{table}[h]
    \resizebox{9cm}{!}{
    \begin{tabular}{ |p{2.8cm}||p{2.8cm}|p{2.8cm}|  }
    
     \hline
     \multicolumn{3}{|c|}{Coefficients of The CO Absorption Lines Equation (equation \ref{eq:CO_DiffPeturbEnrg})} \\
     \hline
     Parameter& Value(\%) & Error (Relative Error) \\ \hline
    $w_{e}[cm^{-1}]$    &$2159.086
    $  &$0.068
    $ ($0.0031\%$) \\
    $x_{e}[1]$  &$0.007315
    $  &$1.6E-05
    % Gadi said E notation is unclear, we should use 10^{??} instead RE: we should maybe ask gary what he prefers. 
    $  ($0.22\%$) \\
    $B_{e}[cm^{-1}]$  &$1.9290
    $     & $0.0090
    $ ($0.47\%$) \\ 
    $\alpha$ [$cm^{-1}$] & $0.01800
    $ &$1.5E-04$ ($0.83\%$) \\
    $D_{v}$ [$cm^{-1}$] & $2.4E-06
    $ & $3.3E-06$ ($140\%$)
     \\ \hline
    \end{tabular}
    }
    \caption{The coefficients from equation \ref{eq:CO_DiffPeturbEnrg} of the CO absorption lines extracted using the fit parameters}
    \label{COcoeftable}
\end{table}

\subsection{Cigar Smoke's Chemical Makeup Analysis}
By looking at the different peak placements and shapes in the IR absorption spectrum of each cigar's smoke and comparing them to the peaks of chemicals with known peaks in the area from the literature\cite{NISTwebook}, 
% Did you only use that one nist website? RE: are you asking me? if so the answer is no lol
the chemical makeup was found. To aid in this endeavor, past research
% Refrence to papers
on the subject was consulted (See \cite{IRCEllSMOKE} and \cite{FTIRSPECTRAOFSMOKE} in bibliography). The LM00 (as opposed to plain "LM") cigar has the widest range of wave numbers, as such, the LM00 cigar smoke's IR spectrum is shown in Figures \ref{fig:DiverseLM} and \ref{fig:COCO2LM}. From now on only "(Brand)'s IR spectrum" is written to be understood as "The (Brand) cigar smoke's IR spectrum". Other IR spectra are quite similar, Galouis and Canadian have the exact same IR spectrum, down to the noise, thus, only Galouis results are shown. NEXT and Marlborro's IR spectra are only between $2000\text{cm}^{-1}$ and $2400\text{cm}^{-1}$ and LM's is between $2000\text{cm}^{-1}$ and $2520\text{cm}^{-1}$, therefore, many substances cannot be identified and quantified.
\begin{figure}
    \includegraphics[width=\linewidth]{Images/Colored Cigar Spectrums/LM00_Diverse_Spectrum.png} 
    \caption{IR Spectrum from $550 \text{cm}^{-1}$ to $1999\text{cm}^{-1}$ of LM cigar smoke with color-coded regions corresponding to the identifying peaks of each substance.}
    \label{fig:DiverseLM}
\end{figure}

\begin{figure}
    \includegraphics[width=\linewidth]{Images/Colored Cigar Spectrums/LM00_CO-CO2_Spectrum.png} 
    \caption{IR Spectrum from $1999 \text{cm}^{-1}$ to $4000 \text{cm}^{-1}$ of LM cigar smoke with color-coded regions corresponding to the identifying peaks of each substance.}
    \label{fig:COCO2LM}
\end{figure}
Note the bumpy area from $3500 \text{cm}^{-1}$ to $4000 \text{cm}^{-1}$, $\text{H}_{2}\text{O}$ and $\text{CO}_{2}$ have been identified in lower wave numbers and this is a mix of the two \cite{FTIRSPECTRAOFSMOKE}.
The area under each substance region was calculated using trapezoidal rule \cite{numerical} 
% reference wikipedia or some paper that shows the method RE: no need.
and is taken as a proxy for substance amount. This metric does not allow for comparison between substances but does allow for comparison between cigar brands.
The identifying regions for each substance and area for each brand are shown in Tables \ref{tbl:CigarSubstanceRegions} and \ref{tbl:CigarSubstanceAreas}. % ref 

\begin{table}[h]
    \begin{tabular}{ |p{2.3cm}|p{3cm}|  }
     \hline
     \multicolumn{2}{|c|}{Peak Ranges For Substances In Cigar Smoke} \\ \hline
     Substance & Range\\ \hline
     CO & 2028-2238 $\text{cm}^{-1}$\\ \hline
     $\text{CO}_{2}$ & 604-705 $\text{cm}^{-1}$\newline 717-724 $\text{cm}^{-1}$\newline 2248-2402 $\text{cm}^{-1}$\\ \hline
     HCN & 710-716 $\text{cm}^{-1}$\\ \hline
     Methanol & 1000-1080 $\text{cm}^{-1}$\\ \hline
     Ethylene & 901-997 $\text{cm}^{-1}$\\ \hline
     Isoprene & 889-897 $\text{cm}^{-1}$\\ \hline
     Methane & 1300-1307 $\text{cm}^{-1}$\newline 2820-3200 $\text{cm}^{-1}$\\ \hline
     Acetaldehyde & 1720-1780 $\text{cm}^{-1}$\\ \hline
     $\text{H}_{2}$O & 1325-1994 $\text{cm}^{-1}$\\ \hline
    \end{tabular}
    \caption{Identifying regions for the found substances.}
    \label{tbl:CigarSubstanceRegions}
\end{table}

\begin{table}[h]
    \begin{tabular}{ |p{1.9cm}|p{1.5cm}|p{1.5cm}|p{1.5cm}|p{1.5cm}|  }
     \hline
     \multicolumn{5}{|c|}{Area Under Identifying Regions For Each Cigar Brand} \\ \hline
     Substance & NEXT & Marlborro & Galouis & LM\\ \hline
     CO & 18 & 6.6 & 18 & 22 \\ \hline % is bad with recipts 
     $\text{CO}_{2}$ & 180 & 69 & 170 & 240 \\ \hline
     HCN  & - & - & 0.37 & 0.51 \\ \hline % is bad with recipts 
     Methanol & - & - & 4.5 & 4.3 \\ \hline % Makes cigars less harsh and more tasty
     Ethylene & - & - & 3.1 & 3.2 \\ \hline % Irrelevant
     Isoprene & - & - & 0.28 & 0.35 \\ \hline % Possible Carcinogen and is known to influence reproductive and developmental processes in animals
     Methane & - & - & 64 & 91 \\ \hline % Found nothing that states its bad in cigars
     Acetaldehyde & - & - & - & - \\ \hline % is bad with recipts 
     $\text{H}_{2}$O & - & - & 43 & 51 \\ \hline
    \end{tabular}
    \caption{Area under the identifying regions of each substance in each brand's smoke.}
    \label{tbl:CigarSubstanceAreas}
\end{table}

Every substance was identified based upon its characteristic absorption lines\cite{NISTwebook}\cite{IRCEllSMOKE}. 
% changed 'using papers cited in the references' to '\cite{}...
Some regions contain multiple absorption lines that overlap, thus restricting the ability to identify the substances that make them up, such as the region of $1300-2000\text{cm}^{-1}$ that contains water but also many other substances such as Acetone, Acetaldehyde and Acrolein \cite{FTIRSPECTRAOFSMOKE}, which is why Acetaldehyde's total absorption could not be estimated. 
NEXT and Marlborro's IR spectra are cut short, only CO and $\text{CO}_{2}$ absorption lines were identified (see Figure \ref{fig:COCO2NEXT}).
As CO and $\text{CO}_{2}$ are present in all smoke spectra a comparison between the different brands is possible. Note that $\text{CO}_{2}$ has absorption lines outside the region $1999-2402 \text{cm}^{-1}$ and thus outside NEXT and Marlborro's IR spectra. As such, the only valid comparison is in the $1999-2402 \text{cm}^{-1}$ region. The ratios of total absorbance between cigars, which according to \ref{eq:BeerLambert} is also the ratio between concentrations was calculated.

\begin{figure}
    \includegraphics[width=\linewidth]{Images/Colored Cigar Spectrums/NEXT_CO-CO2_Spectrum.png} 
    \caption{IR Spectrum from $1999 \text{cm}^{-1}$ to $2402 \text{cm}^{-1}$ of NEXT cigar smoke with color-coded regions corresponding to the identifying peaks of each substance.}
    \label{fig:COCO2NEXT}
\end{figure}

Calculating the ratios between the total absorbance of $\text{CO}_{2}$ in the region 2248-2402 $\text{cm}^{-1}$ and CO in the region 2028-2238 $\text{cm}^{-1}$ yielded the following ratios (see Table \ref{tbl:CigarCO2RATIOS} and Table \ref{tbl:CigarCORATIOS}). The ratios of $\text{CO}_{2}$ were calculated using outside air as a baseline, whose $\text{CO}_{2}$ total absorption in that region was previously measured. Using the concentration of $\text{CO}_{2}$ in air, $C_{\text{CO}_{2}}=0.041\%$ \cite{NASA}, the concentration of $\text{CO}_{2}$ in the smoke from the cigars was found as well. 

\begin{table}[h]
    \begin{tabular}{ |p{1.9cm}|p{1.5cm}|p{1.5cm}|p{1.5cm}|p{1.5cm}|  }
     \hline
     \multicolumn{5}{|c|}{Ratios of $\text{CO}_{2}$ for Different Cigars with outside air as a baseline} \\ \hline
     Property & NEXT & Marlborro & Galouis & LM\\ \hline
     $\text{CO}_{2}$ Ratio & 120 & 44 & 140 & 100 \\ \hline
     $\text{CO}_{2}$ Concentration &4.9\% & 1.8\%& 5.7\% & 4.1\% \\\hline
    
    \end{tabular}
    \caption{Ratios of $\text{CO}_{2}$ for different cigars with outside air as a baseline.}
    \label{tbl:CigarCO2RATIOS}
\end{table}

The ratios of CO were calculated using the NEXT cigar as the baseline (see Table \ref{tbl:CigarCORATIOS}) since no measurement of CO's IR spectrum in air was made in this experiment.

\begin{table}[h]
    \begin{tabular}{ |p{1.5cm}|p{1.5cm}|p{1.5cm}|  }
     \hline
     \multicolumn{3}{|c|}{Ratios of CO for Different Cigars} \\ \hline
      Marlborro & Galouis & LM\\ \hline
       0.36 & 1 & 1.2 \\ \hline
    \end{tabular}
    \caption{Ratios of $\text{CO}_{2}$ for different cigars with NEXT cigar as a baseline.}
    \label{tbl:CigarCORATIOS}
\end{table}

The remaining ratios of gasses can only be calculated between the Galouis and LM cigars since the other two cigars contain only CO and $\text{CO}_{2}$ absorption lines in their spectrum. These remaining ratios would be referred two in the discussion and can be also calculated directly from Table \ref{tbl:CigarSubstanceAreas}.

\section{Discussion}

% The harmful effects of each of the identified substances will now be listed. These effects may not necessarily occur for the concentrations in a cigar. As the concentrations were not measured, this report will be based on previous literature on the toxicity of cigars and cigarettes.
% Starting from the top, Carbon-Monoxide (CO) can take the place of oxygen in blood transport and negatively affect the heart \cite{SubstanceDangerPaper} and dexterity.
% % Add cite{}
% As stated in PubChem: "The first symptoms of carbon monoxide exposure when carboxyhemoglobin is 15-30$\%$ are generalized, and may include: headache, dizziness, nausea, fatigue, and impaired manual dexterity. Individuals with ischemic heart disease may suffer from chest pain and decreased exercise duration at COHb levels measured from 1$\%$ - 9$\%$". Regular smokers have a COHb level 

\bibliography{Report}

\end{document}